\section{The Current System or Situation}

\subsection{Background, Objectives, and Scope}

Right now the current system uses a traditional check-out system where a store 
staff person scans each item, collects the payment 
and bags the groceries. \newline

The objective is to implement a self-service check-out system in all 
their stores.\newline

In the new system the customer will scan the items and bag them, and they will 
pay for their groceries by inserting a card or currency 
or swiping a credit card. \newline

This would make checkout faster since the self checkout system requires less 
hardware making it easier to install more of these which in return would make 
the process more efficient.

\subsection{Operational Policies and Constraints}
You depend on your personal to use your checkout meaning that there are 
times where the checkouts are idle. \newline

You require two person to operate the checkout (the costumer 
and the cashier). \newline

The cashier must be present during the entire checkout process.

\pagebreak

\subsection{Description of the Current System or Situation}
The system is operating inside a supermarket, it requires a carrier band, a 
screen, a printer, scanner gun, scanner panel and a Point-of-Sale.\newline

This Point-of-sale can take cash, credit card (Which are link with different 
credit card providers system) and also support 
Apple Pay and Google pay.\newline

The cashier needs to use the carrier band to bring the items near her so she 
can use either the scan panel or the scan gun to get the pricing of the item 
and add it to the POS.\newline

The carrier band has a fixed amount of items it can carry at a particular time, 
the speed depends on the expertise of the cashier, the amount of items of the 
customer and the current availability from the 
remove payment providers.\newline

The cost of operation requires to maintain all the hardware and also the 
cashier salary.

\subsection{Modes of operation for the current system or situation}
The regular mode of operation would be that the cashier scans and creates a 
final bill for the customer and the customer paid, but there are some 
cases that the cashier requires to return a transaction or requires to open 
the cash drawer. \newline

In this case the cashier would require to call a supervisor that can overwrite 
this transaction or gain more access to the cash drawer. \newline

\pagebreak

\subsection{User classes and other involved personnel}
In the current system exists there types of user, most of the time the user 
are going to be:

\begin{itemize}
	\item \textbf{Cashier}: Who can control the carrier band, add items to the 
    bill, create the bill and confirm payment.
	\item \textbf{Costumer}: Who manually add items to the band and pay the bill
\end{itemize}

If a situation arises there is another that has access to the system which is 
the supervisor who has access to overwrite instructions created by the cashier.

\subsection{Support environment}
For this system there are three companies that take care of the different 
parts of the system, one company maintains the carrier band, the other takes 
care of the POS and the last one takes care of the payment provider.