\section{The Current System or Situation}

\subsection{Background, Objectives, and Scope}

Right now the current system uses a traditional check-out system where a store 
staff person scans each item, collects the payment 
and bags the groceries. \newline

\noindent 
The objective is to implement a self-service check-out system in all 
their stores.\newline

\noindent 
In the new system the customer will scan the items and bag them, 
and they will pay for their groceries by inserting a card or currency 
or swiping a credit card. \newline

\noindent 
This would make checkout faster since the self checkout system 
requires less hardware making it easier to install more of these which in 
return would make the process more efficient.

\subsection{Operational Policies and Constraints}
\begin{itemize}
	\item You depend on your personal to use your checkout meaning that 
	there are times where the checkouts are idle because you don't have 
	enough staff to operate the system
	\item You require two person to operate the checkout (the costumer 
	and the cashier).
	\item The cashier must be present during the entire checkout process.
	\item Only the cashier can start the checkout process
	\item Only the cashier can cancel the checkout process
	\item Only the cashier can handle the payment process
\end{itemize}

\pagebreak

\subsection{Description of the Current System or Situation}

The system is operating inside a supermarket, it requires a carrier band, a 
screen, a printer, scanner gun, scanner panel and a 
Point-of-Sale system.\newline

\noindent
This Point-of-sale can take cash and has a payment processor system 
attach that accepts credit card and also support mobile payments via 
NFC (Apple Pay, Google pay).\newline

\noindent 
The cashier needs to use the carrier band to bring the items near her so she 
can use either the scan panel or the scan gun to get the pricing of the item 
and add it to the POS.\newline

\noindent 
The carrier band has a fixed amount of items it can carry at a particular time, 
the speed depends on the expertise of the cashier, the amount of items of the 
customer and the current availability from the payments providers.\newline

\noindent 
The cost of operation requires to maintain all the hardware and also the 
cashier salary.

\subsection{Modes of operation for the current system or situation}

The regular mode of operation would be that the cashier scans and creates a 
final bill for the customer and the customer paid, but there are some 
cases that the cashier requires to return a transaction or requires to open 
the cash drawer. \newline

\noindent 
In this case the cashier would require to call a supervisor that can overwrite 
this transaction or gain more access to the cash drawer. \newline

\pagebreak

\subsection{User classes and other involved personnel}
In the current system exists there types of user, most of the time the user 
are going to be:

\begin{itemize}
	\item \textbf{Cashier}: Who can control the carrier band, add items to the 
    bill, create the bill and confirm payment.
	\item \textbf{Costumer}: Who manually add items to the band 
	and pay the bill
	\item \textbf{Supervisor}: Can overwrite actions made by the cashier 
	and also other priviledge activities.
\end{itemize}

\subsection{Support environment}

\noindent 
For this system there are three companies that take care of the different 
parts of the system, one company maintains the carrier band, the other takes 
care of the POS (Point-Of-Sale) system and the last one takes care of 
the payment provider.