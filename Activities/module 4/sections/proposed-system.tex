\section{Concepts for the Proposed System}

\subsection{Background, objectives, and scope}

The new system is a Self-checkout (SCO) system, this enables users to do 
the checkout process without the need of a cashier. \newline

The reason we are doing this is because there are some days that we are 
receiving too many clients and there are not enough cashiers so the lines are 
becoming larger and the clients are getting anxious because it takes them 
too long to pay. \newline

A lot of the machinew in the old system stays idle most of the time, this new 
system would optimize this by not needing a cashier to control the 
checkout process. \newline

The new system would have two types of user:
\begin{itemize}
	\item \textbf{Costumer}: Scans, bag and weight the items, and also pay.
	\item \textbf{Super user}: In case something happens this user can 
    overwrite customer operations and has access to more functionalities.
\end{itemize}

\subsection{Operational policies and constraints}

\subsubsection{Operational policies}
\begin{itemize}
    \item The super user is able to overwrite costumer actions
    \item The super user is able to shutdown the system
    \item The super user is able to the system to sleep
\end{itemize}

\subsubsection{Constraints}
\begin{itemize}
    \item The new system only supports credit/debit cards, NFC/Mobile 
    payments and Magnetic Stripe Cards.
    \item The scaling system only supports up to 60Kg
    \item The POS system needs to run on Windows Operating System.
    \item The POS system requires at least 2 meter of space between systems 
    to enable the best use of space.
    \item The system requires at least one person to be operated and up 
    to 2 (one customer, one super user) running the system at the same time.
\end{itemize}

\subsection{Description of the proposed system}

The system will operate in the store where people do their groceries, 
near the supermarket exit, they are going to be running 
alongside the old system. \newline

This system will require a wired connection for electricity to power the 
scanner, printer, scale and POS system. It will require an internet connection 
to the supermarket server and also requires an internet connection with the 
payment providers. \newline

The system requires a scanner, a scale, a printer, a touchable screen, a 
bagging section and payment interface. \newline

The system needs to be able to read barcodes and QR codes, coupons and also 
need to support promo codes. \newline

Each SCO system consumes 10 watts with all the hardware running at the same 
time, 5 watts on idle and 2 watts on sleep mode, the supermarket 
runs from 6:00AM - 10:00PM (16 Hours). \newline

The cost will depend on how many of the SCO will run at the same time, the 
cost of the electricity and the cost of the internet services, which these 
ones are outside the scope of the project. \newline

The operational risk factor would be that if the scale gets broken the SCO is 
not going to work, and if the system loses internet connection the customers 
won't be able to pay since the system doesn't support cash payments. \newline

Since the system only handles one customer at the time the 
system is really fast.

\subsection{Modes of operation}

\subsubsection{Regular}
In this mode all the systems are ready to use the scanner, screen is at 100 
percent of brightness, the bagging system the payment terminal 
and the scale are working.

\subsubsection{Idle}
The machine dims the screen at 50 percent, the bagging system gets shut 
down, the payment terminal shutdown, the scanner gets shut down.

\subsubsection{Sleep}
The machine sleeps the operating system and all the hardware systems 
gets shut down.

\subsubsection{Training}
All the systems are turn on as regular but the checkout information do 
not get store and the payments don't get send to the payment provider.

\subsubsection{Maintenance}
All hardware systems gets shutdown but the mechanical pieces get unlocked.

\subsection{User classes and other involved personnel}
\textbf{Costumer}: Do the checkout process. These are regular people that 
purchase in the supermarket. They have different backgrounds. \newline

\textbf{Super users}: Overwrite customers actions. This type of users are 
trained personal, they are employees Partial Foods employees, they know how to 
operate the SOC system, they are previous user from the old system. \newline

\textbf{Maintenance}: They do diagnostics on the system and fix possible 
issues. They are engineers from the development team, they have technical 
background and also know how the system interacts with the other parts 
of the company. \newline

Usually customers would interact with super users in case they need help or 
if an issue happens. \newline

Super users interact with maintenance because they know what is the current 
state of the system. \newline

Other type of users that do not interact directly with the SOC systems are the 
payment provider services, they are an attachment to the system but the 
maintenance do not have jurisdiction with the payment terminal. \newline

\subsection{Support environment}
The company will provide support for the software itself but the bagging 
area, the scanner and the SOC hardware will be provided by the 
manufacturing company. \newline

The payment terminal would be provided by the company bank or other third 
party service. \newline

Regarding software the maintenance cycle would be each quarter and would be 3 
years warranty support, for further support a new support agreement would be 
required. \newline

Support only support the software itself not the hardware of the system, for 
hardware maintenance would be the a third party supplier and they offer a 
5 year warranty with yearly maintenance cycle. \newline
