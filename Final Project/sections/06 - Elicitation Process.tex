\section{Elicitation}
\subsection{Context}
For the Personalized Shopping Assistant project, we set up meetings every other 
week with our customers, because of COVID we had to have this meeting in a 
remote setting using Google Meet.

\subsection{Methods}
\subsubsection{Brainstorming}
For the first meeting, our goal as a team was to figure out what is the main 
goal of the project, identify the stakeholders and get an idea of the scope of 
the project. For this project, we use user personas \cite{user_personas} to 
identify the stakeholders, because the customer does not have end-users at 
hand so we have to create imaginary roles as reference. \newline

\noindent Next, we wanted to understand the customer's background, so we know 
how to translate ideas using the customer's language as a reference, this 
the background helps us to create analogies that the client can 
understand. \newline

\noindent To achieve this we set up the meeting in a way that each of the 
people involved in the project introduce themselves and tell us what kind of 
experience they have. \newline

\noindent Once we establish this, we have a baseline of the technical 
vocabulary that we can use during the project. \newline

\noindent We noticed the following by observing the group dynamic:
\begin{itemize}
    \item Samantha and Timothy focus mainly on the business side, they are 
    the ones that guide the conversation and have a better understanding 
    on how the application should behave if a particular edge case appears.
    \item Alex focuses more on user privacy, security, and technical 
    implementation.
    \item Kristin more towards user experience, on how to make the application 
    more approachable towards the users.
\end{itemize}

\noindent We discover this by enabling them to talk freely about what is the 
the goal of the project, and each of them tends to participate when a 
particular subject is discussed. \newline

\pagebreak
\subsubsection{Interviews}
From the second meeting onwards, we take the input from the previous meeting 
and we create a set of questions that we send to the client 2-3 days before 
the next meeting so they have time to discuss. \newline

\noindent We use this to get some clarification about requirements that were 
too vague or to confirm how a particular edge case should be handled. \newline

\noindent These questions help us to stay on the scope during the meeting, 
once we are in the meeting, we review the questions one by one and ask more 
information for clarification, and depending on the use case we move the 
conversations towards how to validate them or what quality 
attributes they would expect from the use case. \newline

\noindent If the client does not have the answer to a particular question we 
skip that question and go to the next one. Once we finish with the questions, 
we backtrack to the questions that they did not have an answer and we start 
brainstorming about possible solutions. \newline

\noindent While we are doing the brainstorming session we validate right 
away, how can we verify those requirements and also we evaluate if those are 
feasible or not, sometimes in the conversations, some edge cases arise and we 
think about possible solutions for those edge cases. \newline

\noindent When the meeting finalizes, we take notes of all the answers and 
start analyzing them. We normally take notes of our meetings with our 
customers and make changes accordingly note by note from what we hear 
in response to them.