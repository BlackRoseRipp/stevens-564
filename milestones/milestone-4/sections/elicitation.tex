\section{Elicitation}
\subsection{Context}
For the personal assistant project we set up meetings every other week with 
our customers, because of COVID we had to have this meeting in a remote 
setting using Google Meet.

\subsection{Methods}
\subsubsection{Brainstorming}
For the first meeting, our goal as a team was to figure out what is the main 
goal of the project, identify the stakeholders and get an idea of the scope of 
the project. For this project the stakeholders where imaginary, they were 
potential users.  \newline

\noindent To achieve this we set up the meeting in a way that each of the 
people involved in the project introduce themselves and tell us what kind of 
experience they have, so we have an understanding of the technical knowledge 
each of them has and also so we can balance the technological terms used 
during the conversation. \begin{flushleft} Once we establish this we have a baseline vocabulary that we used 
during the entire project.\end{flushleft} 

\noindent We try to figure out what kind of role they are fulfilling in the 
project, one of them responds better to businesses wise questions, another one 
has technical expertise regarding architectural decision, one is good at 
identifying edge cases and have creative ways to solve problems and the other 
has more technical knowledge regarding security. \newline

\noindent We discover this by enabling them to talk freely about what is the 
goal of the project, and each of them focus on a particular area of the 
project, when a conversation about a particular subject takes place one of 
them usually jumps in. \newline

\noindent In the following meetings once we confirm the requirements from the 
previous meeting we have time to brainstorm about the possible solutions and 
implementations of the feature they would like to include in the project.

\subsubsection{Interviews}
\begin{flushleft}
From the second interview onwards, we take the input from the previous meeting 
and we create a set of questions that we send to the client 2-3 days before 
the meeting so they have time to discuss. \newline

\noindent These questions are created mostly to answer some questions that 
arise while we were analysing the information the client provided in previous 
meetings or to provide some clarification either use case or 
requirement validation.  \newline

\noindent Not only that but these questions help us to stay on scope during 
the meeting, once we are in the meeting we start reviewing the questions one 
by one and ask more information for clarification, and depending on the use 
case we move the conversations towards how to validate them or what quality 
attributes they would expect from the use case. \newline

\noindent If the client does not have the answer to a particular question we 
skip that question and go to the next one, so we can use the time more 
efficiently. Once we finish with the questions we backtrack to the questions 
that they did not have an answer and we start brainstorming about 
possible solutions.  \newline

\noindent While we are doing this brainstorming session we validate right 
away, how can we verify those requirements and also we evaluate if those are 
feasible or not, sometimes in the conversations some edge cases arise and we 
think about possible solutions for those edge cases. \newline

\noindent Once the meeting ends we take notes of all the answers they provided 
and we start analyzing them.
\end{flushleft}