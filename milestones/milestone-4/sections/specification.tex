\section{Specification}
From the conversation that we have with our customers, they have particular 
keywords that they use that help determine what kind of requirements they 
are talking about. \newline

\noindent If they use terms like market share, revenue, sales or terms that 
are associated with money, we know that they are business requirements. \newline

\noindent If the sentences include a functionality or a feature or something 
that the system should be able to do is a user requirement, and during the 
meeting we try to decompose those into what are the inputs and the expected 
outputs of that functionality so we can use them later for validation and to 
write the system requirements properly. \newline

\noindent When we are talking about a particular feature, if the customer 
specifies how it should be a particular function we categorize those as 
functional requirements. \newline

\noindent In this project they have two main focuses, speed and security, 
anytime they mention the word fast, or private data, we know that they are 
talking about quality attributes. \newline

\noindent During the interviews we try to push them to figure out the system 
edge cases, and also we ask them to imagine how the application should work 
under certain conditions, this would be our constraints. \newline

\noindent To figure out how many systems are we building, we use the previous 
exercise to imagine how the application would work, the most frequent words 
they use to describe the system is browser and apps, so we ask for 
clarification on which browser (Mobile or Desktop) and which 
platform (iOS/Android) are they talking about, this ones would be ours 
external interface requirements. \newline

\noindent In general, during the meeting anytime one of the customers uses 
words like standard or normal we ask them to clarify those to avoid ambiguity. 
When they use terms like, perhaps, maybe, could, sometimes, we move the 
conversation in a way that they can evaluate all the possible cases and decide 
on one implementation.  \newline

\noindent If they provide some type of metric either distance or time, we ask 
them to specify the unit and the values, so we can provide specific 
requirements.  \newline

\noindent If they offer a solution to a problem, we ask them how the 
implementation would work, what would be the required steps, the logic behind 
it, how to manage the edge cases, what kind of data are we storing right now 
versus what kind of data we would need to track to achieve the solution, 
sometimes the solution is too ambiguous and the customers decide that they do 
not understand the solution well enough to be added it to the requirements, 
other times that solution creates new requirements and also changes 
existing ones. \newline
